\documentclass[a4paper, 13pt]{article}

%\usepackage{refcheck}
\usepackage[colorlinks = true,
            linkcolor = blue,
            urlcolor  = blue,
            citecolor = blue,
            anchorcolor = blue]{hyperref}
            \usepackage{xcolor}
\makeatletter
\Hy@AtBeginDocument{%
  \def\@pdfborder{0 0 1}% Overrides border definition set with colorlinks=true
  \def\@pdfborderstyle{/S/U/W 1}% Overrides border style set with colorlinks=true
                                % Hyperlink border style will be underline of width 1pt
}
\makeatother
\begin{document}
\title{Teaching Service}
\author{Omar Mohsen
%\thanks{
%Paris-Cité University, Paris, France
%\vskip-2pt E-mail: \href{mailto:omar.mohsen.fr@gmail.com}{\texttt{omar.mohsen.fr@gmail.com}}}
}
\date{Last updated on \today}
\maketitle
% 19K35 Kasparov theory ($KK$-theory)
% 19K56 Index theory
% 22A22 Topological groupoids (including differentiable and Lie groupoids)
% 22A25 Representations of general topological groups and semigroups
% 22E25 Nilpotent and solvable Lie groups
% 46L87 Noncommutative differential geometry
% 46L45 Decomposition theory for $C^*$-algebras
% 46L80 $K$-theory and operator algebras (including cyclic theory)
% 46L87 Noncommutative differential geometry
% 47G30 Pseudodifferential operators	
% 53C12 Foliations (differential geometric aspects)
% 53C29 Issues of holonomy
% 58J22 Exotic index theories
% 58J40 Pseudodifferential and Fourier integral operators on manifolds
% 53R30 Foliations; geometric theory
% 58B34 Noncommutative geometry (�  la Connes)
% 58H05 Pseudogroups and differentiable groupoids 
% 93B18 Linearizations
%\section*{2023}
%\begin{enumerate}
%\item May, Oberwolfach, Germany. Hypoelliptic Operators in Geometry. (Still not clear if I will give a talk or not)
%\end{enumerate}
\section*{2025-2026}
\begin{center}
\begin{tabular}{ |c|c|c|c|c| } 
 \hline
 Année & Semestre & Intitulé & Type & Heures (equiv en TD)\\ \hline
   L1 & S1 & MI1: Algèbre élémentaire I  & Cours/TD & 45\\
    L1 & S1 & Supplément de MI1  & Organisation & 3\\
    L3 & S5 & Topologie & TD & 18\\ 
        L3 & S5 & Topologie & Cours & 18\\
        L2 & S4 & Algèbre bilinéaire et séries de fonctions & TD & 72\\
        M1 & S2 & Topologie Algèbrique & Cours & 36\\ 
   \hline
\end{tabular}
\end{center}
\section*{2024-2025}
\begin{center}
\begin{tabular}{ |c|c|c|c|c| } 
 \hline
 Année & Semestre & Intitulé & Type & Heures (equiv en TD)\\ \hline
   M1 & S2 & Géométrie & TD & 60\\ 
 L3 & S5 & Oraux & TD & 24\\
  L2 & S4 & Oraux & TD & 12 \\
 M1 & S2 & TER (binôme) & Stage & 4\\
  M1 & S2 & TER (trinôme) & Stage & 5\\
    &  & IHES & Délégation & 96 \\  
   \hline
\end{tabular}
\end{center}
\section*{2023-2024}
\begin{center}
\begin{tabular}{ |c|c|c|c|c| } 
 \hline
 Année & Semestre & Intitulé & Type & Heures (equiv en TD)\\ \hline
 L2 & S4 & Analyse de Fourier pour la physique & TD & 47\\ 
  M1 & S2 & Géométrie & TD & 60\\ 
 L3 & S5 & Oraux & TD & 24\\
  L2 & S4 & Oraux & TD & 12 \\
   L0 & S1 & PCSO Géométrie & Cours & 45\\
    M1 & S2 & TER (binôme) & Stage & 4\\
 M1 & S2 & TER (single) & Stage & 3\\
  M1 & S2 & TER (single) & Stage & 3\\
   \hline
\end{tabular}
\end{center}
\newpage
\section*{2022-2023}
\begin{center}
\begin{tabular}{ |c|c|c|c|c| } 
 \hline
 Année & Semestre & Intitulé & Type & Heures (equiv en TD)\\ \hline
 L2 & S4 & Analyse de Fourier pour la physique & TD & 45\\ 
 L3 & S5 & Calcul différential et géométrie & TD & 24\\
 L3 & S5 & Oraux & TD & 24\\
  L2 & S4 & Oraux & TD & 24 \\
   L3 & S5 & Intégration & TD & 24\\
 M1 & S2 & TER (single) & Stage & 3\\
  &  & Nouveau MCF & Décharge & 42 \\  
   \hline
\end{tabular}
\end{center}
\section*{2021-2022}
\begin{center}
\begin{tabular}{ |c|c|c|c|c| } 
 \hline
 Année & Semestre & Intitulé & Type & Heures (equiv en TD)\\ \hline
 L2 & S4 & Analyse de Fourier pour la physique & TD & 45\\ 
 L3 & S5 & Calcul différential et géométrie & TD & 24\\
 L3 & S5 & Oraux & TD & 24\\
  L2 & S4 & Oraux & TD & 24 \\
 L3 & S2 & Algèbre Linéaire & TD & 24\\
 L3 & S5 & Intégration & TD & 24\\
 L3 & S6 & TER (trinôme) & Stage & 5\\
  &  & Nouveau MCF & Décharge & 42 \\  
   \hline
\end{tabular}
\end{center}
%\section*{2018}
%Seminar Bismut
%4. 05.2020, séminaire de l’équipe d’algèbres d’opérateurs, Gottingen, Allemagne.

\end{document}