%------------------------------------------------------------------------------
% CV in Latex
% Author : Charles Rambo
% Based off of: https://github.com/sb2nov/resume and Jake's Resume on Overleaf
% Most recently updated version may be found at https://github.com/fizixmastr 
% License : MIT
%------------------------------------------------------------------------------

\documentclass[A4,11pt]{article}
%\documentclass[letterpaper,11pt]{article} %For use in US
\usepackage{latexsym}
\usepackage[empty]{fullpage}
\usepackage{titlesec}
\usepackage{marvosym}
\usepackage[usenames,dvipsnames]{color}
\usepackage{verbatim}
\usepackage{enumitem}
\usepackage[hidelinks]{hyperref}
\usepackage[english]{babel}
\usepackage{tabularx}
\usepackage{tikz}
\input{glyphtounicode}
\usepackage{latexsym}
\usepackage[empty]{fullpage}
\usepackage{titlesec}
\usepackage{marvosym}
\usepackage[usenames,dvipsnames]{color}
\usepackage{verbatim}
\usepackage{enumitem}
\usepackage[hidelinks]{hyperref}
\usepackage{fancyhdr}
\usepackage[english]{babel}
\usepackage{tabularx}

% my added package for map marker
\usepackage{fontawesome}


\input{glyphtounicode}

\begin{comment}
I am by no means a professional when it comes to the CV's/resumes, I have
received various trainings on how to write a CV and resume from my high 
school, as well as the Austin College and University of Eastern Finland's
career counseling departments. As I intend to share my CV as a template, I 
feel that it is my responsibility to provide explanations of my work.
\end{comment}


%-----FONT OPTIONS-------------------------------------------------------------
\begin{comment}
The font of the document will impact not just how readable it is, but how it is
perceived. In the "The Craft of Scientific Writing" by Michael Alley, shares a
common fonts for publication as well as their use. I have chosen to use
Palatino for its legibility, some others are given below. There is far too much
about typography to discus here. Note: serif fonts have short projecting
strokes, sans-serif fonts are sans (without) these strokes.
\end{comment}


% serif
 \usepackage{palatino}
% \usepackage{times} %This is the default as well
% \usepackage{charter}

% sans-serif
% \usepackage{helvet}
% \usepackage[sfdefault]{noto-sans}
% \usepackage[default]{sourcesanspro}

%-----PAGE SETUP---------------------------------------------------------------

% Adjust margins
\addtolength{\oddsidemargin}{-1cm}
\addtolength{\evensidemargin}{-1cm}
\addtolength{\textwidth}{2cm}
\addtolength{\topmargin}{-1cm}
\addtolength{\textheight}{2cm}

% Margins for US Letter size
%\addtolength{\oddsidemargin}{-0.5in}
%\addtolength{\evensidemargin}{-0.5in}
%\addtolength{\textwidth}{1in}
%\addtolength{\topmargin}{-.5in}
%\addtolength{\textheight}{1.0in}

\urlstyle{same}

\raggedbottom
\raggedright
\setlength{\tabcolsep}{0cm}

% Sections formatting
\titleformat{\section}{
  \vspace{-4pt}\scshape\raggedright\large
}{}{0em}{}[\color{black}\titlerule \vspace{-5pt}]

% Ensure that .pdf is machine readable/ATS parsable
\pdfgentounicode=1

%-----CUSTOM COMMANDS FOR FORMATTING SECTIONS----------------------------------
\newcommand{\CVItem}[1]{
  \item\small{
    {#1 \vspace{-2pt}}
  }
}

\newcommand{\CVSubheading}[4]{
  \vspace{-2pt}\item
    \begin{tabular*}{0.97\textwidth}[t]{l@{\extracolsep{\fill}}r}
      \textbf{#1} & #2 \\
      \small#3 & \small #4 \\
    \end{tabular*}\vspace{-7pt}
}
\newcommand{\medi}[3]{
  \vspace{-2pt}\item
    \begin{tabular*}{0.97\textwidth}[t]{l@{\extracolsep{\fill}}r}
      \textbf{#2} & #1 \\
      \small#3 \\
    \end{tabular*}\vspace{-7pt}
}
\newcommand{\CVSubheadingshort}[2]{
  \vspace{-2pt}\item
    \begin{tabular*}{0.97\textwidth}[t]{l@{\extracolsep{\fill}}r}
      \textbf{#1} & #2 \\
    \end{tabular*}\vspace{-7pt}
}
\newcommand{\CVSubSubheading}[2]{
    \item
    \begin{tabular*}{0.97\textwidth}{l@{\extracolsep{\fill}}r}
      \text{\small#1} & \text{\small #2} \\
    \end{tabular*}\vspace{-7pt}
}

\newcommand{\CVSubItem}[1]{\CVItem{#1}\vspace{-4pt}}

\renewcommand\labelitemii{$\vcenter{\hbox{\tiny$\bullet$}}$}

\newcommand{\CVSubHeadingListStart}{\begin{itemize}[leftmargin=0.5cm, label={}]}
% \newcommand{\resumeSubHeadingListStart}{\begin{itemize}[leftmargin=0.15in, label={}]} % Uncomment for US
\newcommand{\CVSubHeadingListEnd}{\end{itemize}}
\newcommand{\CVItemListStart}{\begin{itemize}}
\newcommand{\CVItemListEnd}{\end{itemize}\vspace{-5pt}}

%------------------------------------------------------------------------------
% CV STARTS HERE  %
%------------------------------------------------------------------------------
\begin{document}

%-----HEADING------------------------------------------------------------------
\begin{comment}
In Europe it is common to include a picture of ones self in the CV. Select
which heading appropriate for the document you are creating.
\end{comment}

\begin{minipage}[c]{0.05\textwidth}
\-\
\end{minipage}
\begin{minipage}[c]{0.2\textwidth}
\begin{tikzpicture}
    \clip (0,0) circle (1.75cm);
    \node at (0,-.3) {\includegraphics[width = 5cm]{moi}}; 
    % if necessary the picture may be moved by changing the at (coordinates)
    % width defines the 'zoom' of the picture
\end{tikzpicture}
\hfill\vline\hfill
\end{minipage}
\begin{minipage}[c]{0.6\textwidth}
    \textbf{\Huge \scshape{Omar Mohsen}} \\ \vspace{1pt} 
    % \scshape sets small capital letters, remove if desired
    \small{\faPhone +33 7 81 32 99 93} \\\small{01 Mars 1996 $|$ French}\\
    \href{mailto:omar.mohsen.fr@gmail.com}{\underline{\faEnvelope\thinspace omar.mohsen.fr@gmail.com}} \\
   \href{mailto:omar.mohsen@universite-paris-saclay.fr}{\underline{\faEnvelope\thinspace omar.mohsen@universite-paris-saclay.fr}}\\ \href{https://sites.google.com/view/omar-mohsen-webpage/home}{\underline{\faGoogle\thinspace https://sites.google.com/view/omar-mohsen-webpage/home}} \\
\end{minipage}

% Without picture
%\begin{center}
%    \textbf{\Huge \scshape Charles Rambo} \\ \vspace{1pt} %\scshape sets small capital letters, remove if desired
%    \small +1 123-456-7890 $|$ 
%    \href{mailto:you@provider.com}{\underline{you@provider.com}} $|$\\
%    % Be sure to use a professional *personal* email address
%    \href{https://linkedin.com/in/your-name-here}{\underline{linkedin.com/in/charles-rambo}} $|$
%    % you should adjust you linked in profile name to be professional and recognizable
%    \href{https://github.com/fizixmastr}{\underline{github.com/fizixmastr}}
%\end{center}


%-----EDUCATION----------------------------------------------------------------
\section{Scientific Career}
  \CVSubHeadingListStart
\CVSubheading
      {{Maître de conférence (research/teaching position)}}{September. 2021 -- Current}
      {\href{https://www.imo.universite-paris-saclay.fr/fr/}{\underline{University of Paris-Saclay}}}{France}   
    \CVSubheading
      {{Postdoc}}{September. 2019 -- August 2021}
      {University of Muenster}{Germany}
        \CVSubheading
      {{ATER (Teaching Position)}}{October 2018 -- August 2019}
      {Paris Diderot university}  {France}
  \CVSubHeadingListEnd
\section{Education}
  \CVSubHeadingListStart
    \CVSubheading
      {{PhD $|$ \emph{\small{Thesis defended in October 2018 under the direction of \href{https://webusers.imj-prg.fr/~georges.skandalis/}{\underline{G. Skandalis}}}}}}{October 2015 -- October 2018}
      {Sorbonne Paris Cité. More details \href{https://theses.fr/2018USPCC200}{\underline{here}}}{France}
    \CVSubheading
      {{ENS Diplome}}{2012 -- 2015}
      {École normale supérieure de Paris}{France}
    \CVSubheading
      {Master}{2013 -- 2015}
      {University of Paris-Saclay}{France}
  \CVSubHeadingListEnd
  \iffalse

\section{Published and accepted for publication articles}
  \CVSubHeadingListStart
  \CVSubheading
  {Higher order derivations on C*-algebras and applications to smooth functional calculus}{2024}
   {Journal of functional analysis}{}  
   \CVSubheading
   {Tangent groupoid and tangent cones in sub-Riemannian geometry}{2024}
    {Duke Mathematical Journal}{}  
    \CVSubheading
    {Witten deformation using Lie groupoids}{2021}
    {Advances in Mathematics}{}  

    \CVSubheading
      {The convolution algebra of Schwarz kernels on a singular foliation}{2020}{Journal of Operator theory}{with Androulidakis and Yuncken}
    \CVSubheading
      {On the deformation groupoid of the inhomogeneous pseudo-differential calculus}{2020}
      {Bulletin of the London Mathematical society}{}
  \CVSubheading
      {Index theorem for inhomogeneous hypoelliptic differential operators}{2020}
      {Muenster Journal of Mathematics}{}
      \CVSubheading
      {Chern Simons invariants in $K\!K$ theory}{2018}
      {Journal of functional analysis}{}  
    
  \CVSubHeadingListEnd

%-----PROJECTS AND RESEARCH----------------------------------------------------
\begin{comment}
Ideally the title of the work should speak for what it is. However if you feel
like you should explain more about why the project is applicable to this job,
use item list as is shown in the work experience section.
\end{comment}

\section{Prepublication Articles}
  \CVSubHeadingListStart
%    \CVSubheading
%      {Title of Work}{When it was done}
%      {Institution you worked with}{unused}
\CVSubheading
{Remarks on a groupoid approach to the Wodzicki residue and the Kontsevich-Vishik trace}{2023}
 {}{}  
  \CVSubheading
   {On the index of maximally hypoelliptic differential operators}{2022}
    {}{}
 \CVSubheading
      {A pseudodifferential calculus for maximally hypoelliptic operators and the Helffer-Nourrigat conjecture}{}
      {with Androulidakis and Yuncken 2022}{}   
 \CVSubheading
   {Blow-up groupoid of singular foliations}{2022}
    {}{}    
  
  \CVSubHeadingListEnd
  \fi
  \section{Scientific Distinctions}
  \CVSubHeadingListStart
%    \CVSubheading
%      {What}{When}
%      {School}{Where}
\CVSubheadingshort
{Our work on Helffer-Nourrigat's conjecture will be presented in a Bourbaki seminar by Debord}{2025}
{}{}
\CVSubheadingshort
{ \href{https://www.college-de-france.fr/fr/personne/omar-mohsen}{\underline{Prix Peccot from Collège de France}}}{2024}
{}{}
   \CVSubheadingshort
     {Prix Jacques Herbrand from the french academy of sciences}{2024}
     {}{}

      \CVSubHeadingListEnd
 
\section{Grants}
  \CVSubHeadingListStart
%    \CVSubheading
%      {What}{When}
%      {School}{Where}
\CVSubheading
{ANR}{2023}
{\href{https://anr.fr/Project-ANR-23-CE40-0016}{\underline{Part of the ANR team OPART}}}{France}
    \CVSubheading
      {Région Ile de France, FSMP}{2015-2018}
      {PhD Grant}{France}
      \CVSubheading
      {Fondation Sciences Mathématiques de Paris (FSMP)}{2013-2015}
      {Scholarship}{France}
       \CVSubheading
      {Fondation Mathématiques Jacques Hadamard (FMJH)}{2012}
      {Scholarship}{France}
  \CVSubHeadingListEnd
  \section{Supervision}
  \CVSubHeadingListStart
%    \CVSubheading
%      {What}{When}
%      {School}{Where}
\CVSubheadingshort
{Julie Capron, Quentin Casella, Oleksii Shulga, Enzo Tanguide}{2024 M1}
\CVSubheadingshort
{Moudrik Chamoux}{2024 M2}
    \CVSubheadingshort
      {Anatole Dedecker}{2023 M1}
      \CVSubheadingshort
      {Matiss Brunel, Quentin Giton, Flore le Roux}{2022 L3}

  \CVSubHeadingListEnd
  \section{Scientific Evaluation}
  I was referee for the following journals: Advances in Mathematics, Annales de l'Institut Fourier, Annales Henri Lebesgue, Astérisque, Bulletin des sciences mathématiques, Communications in Partial Differential Equations, Comptes rendus de l'académie des sciences, Crelle journal, Journal of Differential Equations, Journal of Geometric analysis, Journal of Geometry and Physics, Journal of noncommutative geometry, Math. Ann.  , Muenster journal of mathematics, Pacific Journal of Mathematics
\iffalse
\section{Talks}
  \CVSubHeadingListStart
%    \CVSubheading
%      {What}{When}
%      {School}{Where}
   \medi{04/2021}{Generalisations of Hörmander's sum of squares, parametricies and index theorem}{Global NCG Seminar (Asia-Pacific)}
\medi{03/2021}{Remarques sur la $C^*$-algebre d'une feuilletage singulière }{Séminaire d'algèbres d'opérateurs, université de Clermont-Auvergne}
\medi{03/2021}{Dirac operator on singular foliations}{Séminaire d'algèbres d'opérateurs, Paris Diderot}
\medi{04/2020}{Index theorem for inhomogeneous hypoelliptic differential operators}{Global NCG Seminar, US}
\medi{04/2020}{Index theorem for inhomogeneous hypoelliptic differential operators}{Séminaire à l'université de Göttingen}
\medi{04/2020}{Index theorem for inhomogeneous hypoelliptic differential operators}{Séminaire à l'université de Muenster}
\medi{12/2019}{Index theorem for inhomogeneous hypoelliptic differential operators}{Conférènce à Paris Diderot}
\medi{10/2019}{Witten deformation}{Séminaire à l'université de Muenster}
\medi{06/2019}{Inhomogeneous hypoelliptic differential operators}{Conférence à l'occasion de l'anniversaire de Kasparov, Kyoto, Japan}
\medi{05/2019}{Witten deformation and Lie groupoids}{Conférence à l'honneur de Jean Renault, Orléans, France}
\medi{04/2019}{Witten deformation}{Noncommutative Geometry Festival 2019, Washington University en St. Louis}
\medi{02/2019}{Deformation groupoids.}{Séminaire d'algèbres d'opérateurs, université de Clermont-Auvergne}
\medi{01/2019}{Deformation groupoids.}{Séminaire d'algèbres d'opérateurs, université d'Orléans}
\medi{12/2018}{Inhomogeneous pseudo-differential calculus}{CIMPA School on Noncommutative Geometry and Index Theory}
\medi{10/2018}{Deformation groupoids.}{Séminaire d'Analyse Harmonique non Commutative, université de Caen, France}
	\medi{01/2018}{Variétés de Carnot (filtrées) et calcul pseudodifférentiel associé.}{Groupe de travail Opérateur de Dirac, Orsay, France}
	\medi{01/2018}{On Carnot manifolds and Heisenberg calculus.}{Séminaire d'algèbres d'opérateurs, université Paris Diderot, France}
	\medi{11/2017}{Chern Simons invariants in $K\! K$ theory.}{Groupe de travail à Saint-Flour, France}
  \CVSubHeadingListEnd
  \fi
\section{Scientific and administrative responsibilities}
 \CVSubHeadingListStart
 \CVSubheadingshort
      {\href{https://theses.fr/s279397}{\underline{Member of PhD defense of Clement Cren under the direction of J.-M. Lescure}}}{2023 France}
  \CVSubheadingshort
      {Summer school YMC$^*$A, Organization member}{2021, Germany}
      \CVSubheadingshort
      {PhD students seminar Paris Diderot, Organization member}{2017-2018, France}
          \CVSubheading
      {Work group on Atiyah-Singer index theorem}{2014}
      {I organized a work group in the ENS following « Seminar on Atiyah-Singer index theorem » by S. Palais}{France}
   \CVSubHeadingListEnd
\section{Temporary visiting positions}
   \CVSubHeadingListStart
   \CVSubheadingshort
        {IHES}{Duration: 6 months, in 2024}
     \CVSubHeadingListEnd
  
%-----COMMUNITY INVOLVEMENT----------------------------------------------------
\begin{comment}
\section{Languages}
  \CVSubHeadingListStart
%    \CVSubheading %Example
%      {What you did}{When you worked there}
%      {Who you worked for}{Where they are located}
    \CVSubheadingshort
      {Arabe}{Mother language}
   \CVSubheadingshort
      {English}{Fluent} 
      \CVSubheadingshort
      {French}{Advanced}
  \CVSubHeadingListEnd
\end{comment}
%-----SKILLS-------------------------------------------------------------------
%\begin{comment}
%This section is compressed from the various skills sections that Euro CV
%recommends.
%\end{comment}

\section{Languages}
 \begin{itemize}[leftmargin=0.5cm, label={}]
    \small{\item{
     Arabic (Mother Language), English (Fluent), French (Advanced) \\
     %\textbf{Piano}{: around Grade 10 following the The RCM (Royal Conservatory of Music) grading} \\
    % \textbf{Document Creation}{: Microsoft Office Suite, LaTex, Markdown} \\
    }}
 \end{itemize}
    
%------------------------------------------------------------------------------
\end{document}