%------------------------------------------------------------------------------
% CV in Latex
% Author : Charles Rambo
% Based off of: https://github.com/sb2nov/resume and Jake's Resume on Overleaf
% Most recently updated version may be found at https://github.com/fizixmastr 
% License : MIT
%------------------------------------------------------------------------------

\documentclass[A4,11pt]{article}
%\documentclass[letterpaper,11pt]{article} %For use in US
\usepackage{latexsym}
\usepackage[empty]{fullpage}
\usepackage{titlesec}
\usepackage{marvosym}
\usepackage[usenames,dvipsnames]{color}
\usepackage{verbatim}
\usepackage{enumitem}
\usepackage[hidelinks]{hyperref}
\usepackage[english]{babel}
\usepackage{tabularx}
\usepackage{tikz}
\input{glyphtounicode}
\usepackage{latexsym}
\usepackage[empty]{fullpage}
\usepackage{titlesec}
\usepackage{marvosym}
\usepackage[usenames,dvipsnames]{color}
\usepackage{verbatim}
\usepackage{enumitem}
\usepackage[hidelinks]{hyperref}
\usepackage{fancyhdr}
\usepackage[english]{babel}
\usepackage{tabularx}

% my added package for map marker
\usepackage{fontawesome}


\input{glyphtounicode}



%-----FONT OPTIONS-------------------------------------------------------------
\begin{comment}
The font of the document will impact not just how readable it is, but how it is
perceived. In the "The Craft of Scientific Writing" by Michael Alley, shares a
common fonts for publication as well as their use. I have chosen to use
Palatino for its legibility, some others are given below. There is far too much
about typography to discus here. Note: serif fonts have short projecting
strokes, sans-serif fonts are sans (without) these strokes.
\end{comment}


% serif
 \usepackage{palatino}

\addtolength{\oddsidemargin}{-1cm}
\addtolength{\evensidemargin}{-1cm}
\addtolength{\textwidth}{2cm}
\addtolength{\topmargin}{-1cm}
\addtolength{\textheight}{2cm}



\urlstyle{same}

\raggedbottom
\raggedright
\setlength{\tabcolsep}{0cm}

\titleformat{\section}{
  \vspace{-4pt}\scshape\raggedright\large
}{}{0em}{}[\color{black}\titlerule \vspace{-5pt}]

\pdfgentounicode=1

\newcommand{\CVItem}[1]{
  \item\small{
    {#1 \vspace{-2pt}}
  }
}

\newcommand{\CVSubheading}[4]{
  \vspace{-2pt}\item
    \begin{tabular*}{0.97\textwidth}[t]{l@{\extracolsep{\fill}}r}
      \textbf{#1} & #2 \\
      \small#3 & \small #4 \\
    \end{tabular*}\vspace{-7pt}
}
\newcommand{\medi}[3]{
  \vspace{-2pt}\item
    \begin{tabular*}{0.97\textwidth}[t]{l@{\extracolsep{\fill}}r}
      \textbf{#2} & #1 \\
      \small#3 \\
    \end{tabular*}\vspace{-7pt}
}
\newcommand{\CVSubheadingshort}[2]{
  \vspace{-2pt}\item
    \begin{tabular*}{0.97\textwidth}[t]{l@{\extracolsep{\fill}}r}
      \textbf{#1} & #2 \\
    \end{tabular*}\vspace{-7pt}
}
\newcommand{\CVSubSubheading}[2]{
    \item
    \begin{tabular*}{0.97\textwidth}{l@{\extracolsep{\fill}}r}
      \text{\small#1} & \text{\small #2} \\
    \end{tabular*}\vspace{-7pt}
}

\newcommand{\CVSubItem}[1]{\CVItem{#1}\vspace{-4pt}}

\renewcommand\labelitemii{$\vcenter{\hbox{\tiny$\bullet$}}$}

\newcommand{\CVSubHeadingListStart}{\begin{itemize}[leftmargin=0.5cm, label={}]}
\newcommand{\CVSubHeadingListEnd}{\end{itemize}}
\newcommand{\CVItemListStart}{\begin{itemize}}
\newcommand{\CVItemListEnd}{\end{itemize}\vspace{-5pt}}
\begin{document}

\begin{comment}
\end{comment}

\begin{minipage}[c]{0.05\textwidth}
\-\
\end{minipage}
\begin{minipage}[c]{0.2\textwidth}
\begin{tikzpicture}
    \clip (0,0) circle (1.75cm);
    \node at (0,-.3) {\includegraphics[width = 5cm]{moi}}; 
\end{tikzpicture}
\hfill\vline\hfill
\end{minipage}
\begin{minipage}[c]{0.6\textwidth}
    \textbf{\Huge \scshape{Omar Mohsen}} \\ \vspace{1pt} 
    \small{\faPhone +33 7 81 32 99 93} \\\small{01 Mars 1996 $|$ French}\\
    \href{mailto:omar.mohsen.fr@gmail.com}{\underline{\faEnvelope\thinspace omar.mohsen.fr@gmail.com}} \\
   \href{mailto:omar.mohsen@universite-paris-saclay.fr}{\underline{\faEnvelope\thinspace omar.mohsen@universite-paris-saclay.fr}}\\ \href{https://sites.google.com/view/omar-mohsen-webpage/home}{\underline{\faGoogle\thinspace sites.google.com/view/omar-mohsen-webpage/home}} \\
   \href{https://github.com/OmarMohsenGit}{\underline{\faGithub \thinspace github.com/OmarMohsenGit}}
  \end{minipage}

%-----EDUCATION----------------------------------------------------------------
\section{Scientific Career}
  \CVSubHeadingListStart
\CVSubheading
      {{Maître de conférence (research/teaching position)}}{September. 2021 -- Current}
      {\href{https://www.imo.universite-paris-saclay.fr/fr/}{\underline{University of Paris-Saclay}}}{France}   
    \CVSubheading
      {{Postdoc}}{September. 2019 -- August 2021}
      {University of Muenster}{Germany}
        \CVSubheading
      {{ATER (Teaching Position)}}{October 2018 -- August 2019}
      {Paris Diderot university}  {France}
  \CVSubHeadingListEnd
\section{Education}
  \CVSubHeadingListStart
    \CVSubheading
      {{PhD $|$ \emph{\small{Thesis defended in October 2018 under the direction of \href{https://webusers.imj-prg.fr/~georges.skandalis/}{\underline{G. Skandalis}}}}}}{October 2015 -- October 2018}
      {Sorbonne Paris Cité. More details \href{https://theses.fr/2018USPCC200}{\underline{here}}}{France}
    \CVSubheading
      {{ENS Diplome}}{2012 -- 2015}
      {École normale supérieure de Paris}{France}
    \CVSubheading
      {Master}{2013 -- 2015}
      {University of Paris-Saclay}{France}
  \CVSubHeadingListEnd
  \section{Scientific Distinctions}
\CVSubHeadingListStart
\CVSubheadingshort{Our work on Helffer-Nourrigat's conjecture will be presented in a Bourbaki seminar by \href{https://webusers.imj-prg.fr/~claire.debord/}{\underline{Debord}}}{2025}{}{}
\CVSubheadingshort{\href{https://www.college-de-france.fr/fr/personne/omar-mohsen}{\underline{Prix Peccot from Collège de France}}}{2024}{}{}
   %\CVSubheadingshort{Prix Jacques Herbrand from the french academy of sciences}{2024}{}{}
\CVSubHeadingListEnd
 
\section{Grants}
  \CVSubHeadingListStart
\CVSubheading
{ANR}{2023}
{\href{https://anr.fr/Project-ANR-23-CE40-0016}{\underline{Part of the ANR team OPART}}}{France}
    \CVSubheading
      {Région Ile de France, FSMP}{2015-2018}
      {PhD Grant}{France}
      \CVSubheading
      {Fondation Sciences Mathématiques de Paris (FSMP)}{2013-2015}
      {Scholarship}{France}
       \CVSubheading
      {Fondation Mathématiques Jacques Hadamard (FMJH)}{2012}
      {Scholarship}{France}
  \CVSubHeadingListEnd
  \section{Supervision}
  \CVSubHeadingListStart
  
\CVSubheadingshort
{Julie Capron, Quentin Casella, Oleksii Shulga, Enzo Tanguide}{2024 M1}
\CVSubheadingshort
{Moudrik Chamoux}{2024 M2}
    \CVSubheadingshort
      {Anatole Dedecker}{2023 M1}
      \CVSubheadingshort
      {Matiss Brunel, Quentin Giton, Flore le Roux}{2022 L3}

  \CVSubHeadingListEnd
  \section{Scientific Evaluation}
  I was referee for the following journals: Advances in Mathematics, Annales de l'Institut Fourier, Annales Henri Lebesgue, Astérisque, Bulletin des sciences mathématiques, Communications in Partial Differential Equations, Comptes rendus de l'académie des sciences, Crelle journal, Journal of Differential Equations, Journal of Geometric analysis, Journal of Geometry and Physics, Journal of noncommutative geometry, Math. Ann.  , Muenster journal of mathematics, Pacific Journal of Mathematics
\section{Scientific and administrative responsibilities}
 \CVSubHeadingListStart
 \CVSubheadingshort
      {\href{https://theses.fr/s279397}{\underline{Member of PhD defense of Clement Cren under the direction of J.-M. Lescure}}}{2023 France}
  \CVSubheadingshort
      {Summer school YMC$^*$A, Organization member}{2021, Germany}
      \CVSubheadingshort
      {PhD students seminar Paris Diderot, Organization member}{2017-2018, France}
          \CVSubheading
      {Work group on Atiyah-Singer index theorem}{2014}
      {I organized a work group in the ENS following « Seminar on Atiyah-Singer index theorem » by S. Palais}{France}
   \CVSubHeadingListEnd
\section{Temporary visiting positions}
   \CVSubHeadingListStart
   \CVSubheadingshort
        {IHES}{Duration: 6 months, in 2024}
     \CVSubHeadingListEnd


\section{Languages}
 \begin{itemize}[leftmargin=0.5cm, label={}]
    \small{\item{Arabic (Mother Language), English (Fluent), French (Advanced)}\let\thefootnote\relax\footnotetext{{Last updated on \today}}}
 \end{itemize}
\end{document}